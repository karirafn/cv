\documentclass[11pt,a4paper,sans]{moderncv}

\usepackage[T1]{fontenc}
\usepackage[icelandic]{babel}
\usepackage[utf8]{inputenc}
\usepackage[scale=0.8]{geometry}

\moderncvstyle[black]{casual}

\newcommand{\cvdoublecolumn}[2]
{
	\cvitem[0.75em]{}
	{
		\begin{minipage}[t]{\listdoubleitemcolumnwidth}#1\end{minipage}
		\hfill
		\begin{minipage}[t]{\listdoubleitemcolumnwidth}#2\end{minipage}
	}
}

\newcommand{\cvreference}[5]
{
	\textbf{#1}\newline
	\addresssymbol~#2\newline
	#3\newline
	\emailsymbol~\texttt{\href{mailto:#4}{\nolinkurl{#4}}}\newline
	\mobilesymbol~#5
}

\name{}{Kári Rafn Karlsson}
\address{Eyrarlundur 4}{300 Akranes}{}
\phone[mobile]{695~3348}
\email{karirafn@gmail.com}
\extrainfo{07/02/1985}
\photo[64pt][0.1pt]{picture}
\social[linkedin]{karirafn}
\social[github]{karirafn}

\begin{document}
	
	\maketitle
	
	\section{Starfsreynsla}

        \cventry{2020 -- }{PDM sérfræðingur og forritari}{Skaginn 3X}{}{}{
            % Yfirhönnuður á IIS hýstri innanhús REST API og vefsíðu unnið í .NET 6, SQL og Blazor Server og notar GitHub Actions fyrir CI/CD. Hönnun sjálfvirkra ferla með .NET framework 4.8 og gefið út með PowerShell skriptum.
            \begin{itemize}
                \item Umsjón Autodesk Inventor CAD (\emph{Computer Aided Design}) forrita
                \item Umsjón Autodesk Vault PDM (\emph{Product Data Management}) kerfis
                \item Hönnun, forritun og innleiðing á REST API og Blazor Server vefsíðu í .NET 6 (Core)
                \item CI/CD með GitHub Actions
                \item Hönnun viðbóta fyrir CAD og PDM forrit í .NET Framework 4.8 og PowerShell
                \begin{itemize}
                    \item Sjálfvirkir útreikningar á hráefnisþörf fyrir smíðaða íhluti
                    \item Sjálfvirkur flutningur gagna úr Vault yfir í MS Dynamics NAV
                    \item Sjálfvirk sannreynsla og leiðrétting á hönnunargögnum
                    \item o.fl.
                \end{itemize}
            \end{itemize}
            }

        \cventry{2020 -- }{PDM og skýjalausnaráðgjafi}{Sjálfstætt starfandi}{}{}{
            % Uppsetning og viðhald Autodesk Vault PDM kerfa í Azure með bicep skriptum auk hönnun Autodesk viðbóta í .NET.
            \begin{itemize}
                \item Uppsetning og viðhald kerfa í Azure (VM, VPN, VNET, o.fl.)
                \item Uppsetning, innleiðing og viðhald Autodesk Vault PDM kerfa
                \item Hönnun Autodesk viðbóta í .NET
            \end{itemize}
            }

        \cventry{2018 -- 2020}{Vélahönnuður}{Skaginn 3X}{}{}{
            % Vörueigandi sjálfvirkra plötufrysta, skilgreining innri ferla og flæða, uppsetning og umsjón PDM kerfis, sjálfvirkt eftirlit og flæði gagna úr PDM yfir ERP kerfi, stöðlun íhluta o.fl.
            \begin{itemize}
                \item Vörueigandi sjálfvirkra plötufrysta
                \item Skilgreining innri ferla og flæða
                \item Sjálfvirknivæðing ferla með .NET
                \item Uppsetning, innleiðing og umsjón Autodesk Vault PDM kerfis
                \item Stöðlun íhluta
            \end{itemize}
            }
        
        \cventry{2015 -- 2017}{Vélahönnuður}{VHE}{}{}{
            % Vélahönnun, sjálfvirknivæðing ferla með, hönnun vörusamskipunartóls (e. product configurator) í .NET og uppsetning/umsjón PDM kerfis.
            \begin{itemize}
                \item Vélahönnun
                \item Sjálfvirknivæðing ferla með .NET
                \item Hönnun vörusamskipunartóls (e. product configurator) í .NET
                \item Uppsetning, innleiðing og umsjón Autodesk Vault PDM kerfis
            \end{itemize}
            }
	
    \section{Menntun}

        \cventry{2009 --}{M.Sc. Vélaverkfræði}{Háskólinn í Reykjavík}{}{}{}
        \cventry{2006 -- 2009}{B.Sc. Hátækniverkfræði}{Háskólinn í Reykjavík}{}{}{}
        \cventry{2001 -- 2005}{Náttúrufræðibraut}{Fjölbrautaskóli Vesturlands Akranesi}{}{}{}

	\section{Forritunarhæfileikar}

        \cvitem{Forritunarmál}{
            Mjög gott vald á C\# auk mismikillar reynslu í Java, JavaScript/TypeScript, Matlab, Python, Rust og VB.}

        \cvitem{Hugbúnaður og tól}{
            Yfirgripsmikil reynsla af .NET, ASP.NET, Entity Framework, Blazor, MS SQL Server, Azure og GitHub Actions. Hef notast við WPF og Docker.}
	
    \newpage

	\section{Áhugamál}

        \cvitem{Forritun}{
            Í frítíma mínum hef ég gefið út opinn hugbúnað sem kallast FluentVault (https://github.com/karirafn/fluentvault). Ég kynni mér nýja fídusa í C\# og .NET eða áhugaverðar lausnir auk þess að prófa mig áfram með önnur tungumál. Ég horfi á upptökur frá ráðstefnum sem vekja áhuga minn, t.d. NDC, goto; o.fl.}

        \cvitem{Kraftlyftingar}{
            Ég er með alþjóðleg dómararéttingi í kraftlyftingum og fer erlendis 1-2 sinnum á ári til þess að dæma á alþjóðamótum. Ég var gjaldkeri í stjórn Kraftlyftingasambands Íslands um árabil og var formaður landsliðsnefndar.}
	
    \section{Tungumál}

        \cvitemwithcomment{Íslenska}{Mjög góð}{
            Mjög gott vald á rituðu og töluðu máli.}
        
        \cvitemwithcomment{Enska}{Mjög góð}{
            Nánast sama vald á ensku og íslensku, þ.m.t. tæknimáli.}
        
        \cvitemwithcomment{Danska}{Ágæt}{
            Gott vald á rituðu máli.}
        
        \cvitemwithcomment{Þýska}{Sæmileg}{
            Fjórir áfangar í framhaldsskóla.}

	\section{Meðmælendur}

        \cventry
        {}
        {Trausti Árnason}
        {VP of Product}
        {Controlant}
        {}
        {\emailsymbol trausti.arnason@controlant.com \mobilesymbol 825 8247}

        \cventry
        {}
        {Arnór Freyr Símonarson}
        {Vélahönnuður}
        {Efla}
        {}
        {\emailsymbol afs@efla.is \mobilesymbol 659 5542}

        \cventry
        {}
        {Axel Hreinn Steinþórsson}
        {Eigandi}
        {Eurometal}
        {}
        {\emailsymbol axel@eurometal.is \mobilesymbol 844 5814}

        \cventry
        {}
        {Einir Pálsson}
        {Forritari}
        {Skaginn 3X}
        {}
        {\emailsymbol einir@skaginn3x.com \mobilesymbol 866 6352}
        
        \cventry
        {}
        {Nökkvi Andersen}
        {Verkefnastjóri}
        {Carbfix}
        {}
        {\emailsymbol nokkvi.andersen@or.is \mobilesymbol 899 9489}

\end{document}